% Created 2022-08-24 mié 13:55
% Intended LaTeX compiler: pdflatex
\documentclass[12pt]{article}
\usepackage[utf8]{inputenc}
\usepackage[T1]{fontenc}
\usepackage{graphicx}
\usepackage{grffile}
\usepackage{longtable}
\usepackage{wrapfig}
\usepackage{rotating}
\usepackage[normalem]{ulem}
\usepackage{amsmath}
\usepackage{textcomp}
\usepackage{amssymb}
\usepackage{capt-of}
\usepackage{hyperref}
\usepackage[spanish]{babel}
\usepackage{graphicx,geometry}
\geometry{ a4paper, left=1in, right=1in, top=1in, bottom=1in }
\renewcommand\familydefault{\sfdefault}
\usepackage{sectsty}
\sectionfont{\normalfont\large }
\usepackage{tabularx}
\usepackage{listings}
\lstdefinestyle{mystyle}{
numbers=left,
showspaces=false,
frame=leftline,
showspaces=false,
showstringspaces=false,
showtabs=false,
numberstyle=\tiny,
}
\lstset{
style=mystyle,
literate={á}{{\'a}}1
{é}{{\'e}}1
{í}{{\'{\i}}}1
{ó}{{\'o}}1
{ú}{{\'u}}1
{Á}{{\'A}}1
{É}{{\'E}}1
{Í}{{\'I}}1
{Ó}{{\'O}}1
{Ú}{{\'U}}1
{ü}{{\"u}}1
{Ü}{{\"U}}1
{ñ}{{\~n}}1
{Ñ}{{\~N}}1
{¿}{{?``}}1
{¡}{{!``}}1
}
\makeatletter
\usepackage{fancyhdr}
\pagestyle{fancy}
\usepackage{mdframed}
\BeforeBeginEnvironment{minted}{\begin{mdframed}}
\AfterEndEnvironment{minted}{\end{mdframed}}
\setcounter{secnumdepth}{1}
\author{Luis Eduardo Galindo Amaya (1274895)}
\date{2022-08-24 mié}
\title{Calculadora de Matrices}
\hypersetup{
 pdfauthor={Luis Eduardo Galindo Amaya (1274895)},
 pdftitle={Calculadora de Matrices},
 pdfkeywords={},
 pdfsubject={},
 pdfcreator={Emacs 26.3 (Org mode 9.1.9)}, 
 pdflang={Spanish}}
\begin{document}


\newcommand{\docente}{Manuel Castañón-Puga}
\newcommand{\asignatura}{Herramientas de Desarrollo de Software (40017)}
\newcommand{\semestre}{2022-2}

\newcommand{\miportada}[1]{
	\begin{titlepage}
		\vspace*{0.75in}
		\begin{flushleft}
			\sffamily
			\large #1       \\
			\Huge 
            \@title         \\
			\hrulefill
			\vspace{0.25in} \\
			\Large \@author \\
			\vspace*{\fill}
            \includegraphics[width=\textwidth]{../includes/filler.png} \\
			\vspace*{\fill}
			\large
			\begin{tabular}{|l|l|}
              \hline
			  Asignatura & \asignatura \\
			  Docente    & \docente    \\
			  Fecha      & \@date      \\
              \hline
			\end{tabular}
		\end{flushleft}
	\end{titlepage}
}

\miportada{ Práctica 1 }

\fancyhf{}
\lhead{ \asignatura }
\rhead{ \semestre }
\rfoot{Página \thepage}

\setlength\parindent{0pt}   % eliminar el intentado
\setlength{\parskip}{1.2em}
% \maketitle

\thispagestyle{empty}
\begin{center}
	{\large
		UNIVERSIDAD AUTÓNOMA DE BAJA CALIFORNIA \\
		Facultad de Ciencias Químicas e Ingeniería }
	\vspace{0.25in} \\
	Programa de Ingeniero en Software y Tecnologías Emergentes
\end{center}

\section{Codigo A}
\label{sec:org8543722}
\lipsum[1-1]

\section{Código B}
\label{sec:orgfa1a854}
\lstinputlisting{matrices.c}


\begin{lstlisting}
/ Escriba un código que solicite 2 números y los reste. Desplegar un 1 si el
/ resultado fue negativo o un 0 en caso contrario.  

 INPUT                            / Captura el valor de X
 Store X                          
 INPUT                            / Captura un valor de Y
 Store Y                          
 
 load X                           / Carga el valor de X en el acumulador
 Subt Y                           / Resta el valor de Y a X 
 
                                  / Para ese punto, el valor en AC es 'X-Y'
 
 Skipcond 000                     / si AC es mayor o igual a 0 
 clear                            / el valor en AC se vuelve 0
 
 Skipcond 400                     / si es diferente a 0
 Load verdadero                   / el valor en AC se vuelve 1
 
 Output
 Halt
 
 X, DEC 0
 Y, DEC 0
 verdadero, dec 1
\end{lstlisting}

\begin{lstlisting}
/ ESCRIBA UN CÓDIGO QUE SOLICITE 2 NÚMEROS Y LOS RESTE. 
/ DESPLEGAR UN 1 SI EL RESULTADO FUE NEGATIVO O UN 0 EN CASO 
/ CONTRARIO.  

INPUT                   / CAPTURA EL VALOR DE X
STORE X                          
INPUT                   / CAPTURA UN VALOR DE Y
STORE Y                          

LOAD X                  / CARGA EL VALOR DE X EN EL ACUMULADOR
SUBT Y                  / RESTA EL VALOR DE Y A X 

                        / EL VALOR EN AC ES 'X-Y'

SKIPCOND 000            / SI AC ES MAYOR O IGUAL A 0 
CLEAR                   / EL VALOR EN AC SE VUELVE 0

SKIPCOND 400            / SI ES DIFERENTE A 0
LOAD VERDADERO          / EL VALOR EN AC SE VUELVE 1

OUTPUT
HALT

X, DEC 0
Y, DEC 0
VERDADERO, DEC 1
\end{lstlisting}
\end{document}
